\documentclass[a4paper,14pt]{article}
\usepackage[cp1251]{inputenc}
\usepackage[russian]{babel}
\usepackage[left=2cm,right=2cm,top=2cm,bottom=2cm]{geometry}

\pagestyle{empty}

\renewcommand{\baselinestretch}{1} 

\begin{document}

\begin{itemize}

	\item How to copy a whole line: Just press \texttt{yy} (y and y two times); move where you wish and press \texttt{p} to paste.

        \item Repeat the last action. Press \texttt{.} (dot). 
	
	\item  Shift block right. \begin{enumerate}
		
			\item Move cursor to the beginning of the block;
			\item Press \texttt{v} to select (a block) multiple lines;
			\item Press \texttt{>} or \texttt{<} to move right or left respectively;
			\item If multiple shifts needed, just press \texttt{.}.
		\end{enumerate}


	\item  Insert vertically multiple times. \begin{enumerate}
		\item Move the cursor where you wish;
		\item Press \texttt{Ctrl+v} and select lines where new text will be inserted; 
		\item Press \texttt{Shift+I} to tell Vim that you will insert something new. 
		\item Type anything ... 
		\item Press \texttt{ESC} and the anything will be inserted.
		\end{enumerate}

	\item Using registers. Pressing \texttt{yy} stores current line to 
		default register \texttt{""} or \texttt{"0};
		One can use custom register name to copy and paste text. To store
		current line in a specified register type: \texttt{"qyy}, where \texttt{q}
		is a register name; To paste from the \texttt{q} register type \texttt{"qp}.
		

\end{itemize}



\end{document}
